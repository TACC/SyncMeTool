\documentclass{beamer}

% You can also use a 16:9 aspect ratio:
%\documentclass[aspectratio=169]{beamer}
\usetheme{TACC16}

% It's possible to move the footer to the right:
%\usetheme[rightfooter]{TACC16}

%% page 
%\begin{frame}{}
%  \begin{itemize}
%    \item
%  \end{itemize}
%\end{frame}
%
%% page 
%\begin{frame}[fragile]
%    \frametitle{}
% {\tiny
%    \begin{semiverbatim}
%    \end{semiverbatim}
%}
%  \begin{itemize}
%    \item
%  \end{itemize}
%
%\end{frame}

\begin{document}
\title[SyncMe]{SyncMe and Git Worktrees}
\author{Robert McLay} 
\date{May 1 2024}

% page 1
\frame{\titlepage} 


% page 2
\begin{frame}{Outline}
  \begin{itemize}
    \item This talk is in two parts: SyncMe and Git Worktrees
    \item Part 1: Help when you have too-many git repos
    \item Part 2: Help whne you have more that one active branch
  \end{itemize}
\end{frame}

% page 3
\begin{frame}{Motivation for the syncMe tool }
  \begin{itemize}
    \item Too many repos: 28 on laptop and 22 on Frontera
    \item Examples: my dotfiles (.up), Lmod and XALT
    \item Hard to keep then all up-to-date
    \item Result: syncMe in the SyncMeTool repo
    \item https://github.com/TACC/SyncMeTool.git
  \end{itemize}
\end{frame}

% page 4
\begin{frame}{What does syncMe do?}
  \begin{enumerate}
    \item Uses \$SyncDirPath for were to look for git repos
    \item Can report the status for all those directorys
    \item Can do a pull on those directories.
  \end{enumerate}
\end{frame}

% page 5
\begin{frame}{Install}
  \begin{itemize}
    \item See https://github.com/TACC/SyncMeTool.git for complete details
    \item TLDR:
    \item Install lua and luaposix if necessary
    \item clone repo
    \item Put SyncMeTool/bin in your \$PATH
  \end{itemize}
\end{frame}

% page 6
\begin{frame}{Usage: Specify where to look: \$SyncDirPath}
  \begin{itemize}
    \item Set \$SyncDirPath to be a list of directories to search for
      ``.git``
    \item Paths w/o a leading '/' are relative to \$HOME
  \end{itemize}
\end{frame}

% page 7
\begin{frame}{Usage: syncMe status or syncMe st}
  \begin{itemize}
    \item Loop over directories, search for ``.git'' directory or file
    \item Runs git status in those directories
    \item Removes unnecessary output
    \item Those repos with necessary output are buffered and printed
      at the end.
  \end{itemize}
\end{frame}

% page 8
\begin{frame}{Usage: syncMe pull or syncMe up}
  \begin{itemize}
    \item Loop over directories, search for ``.git'' directory or file
    \item Runs git pull in those directories on origin
    \item Removes unnecessary output
    \item Those repos with necessary output are buffered and printed
      at the end.
  \end{itemize}
\end{frame}


% page 9
\begin{frame}{Demo}
  \begin{itemize}
    \item Show this in practice.
  \end{itemize}
\end{frame}

% page 10
\begin{frame}{Part 2: Git Worktrees}
  \begin{itemize}
    \item Two projects: Lmod and XALT have more than one active branch
    \item Switching branches in same directory tree ususlly unhelpful
    \item Especially running tests in two different branches
    \item Could use multiple clones, but I have lost track of them.
  \end{itemize}
\end{frame}

% page 11
\begin{frame}{What are git worktree?}
  \begin{itemize}
    \item They are part of git but not well known.
    \item They allow for multiple source directories from one clone
    \item You can leave branches in incomplete state and switch to a
      different branch 
    \item No need to use ``git stash``
  \end{itemize}
\end{frame}

% page 12
\begin{frame}{Demo}
  \begin{itemize}
    \item Show this in practice.
  \end{itemize}
\end{frame}




\end{document}
